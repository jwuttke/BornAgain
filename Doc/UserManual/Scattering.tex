%%%%%%%%%%%%%%%%%%%%%%%%%%%%%%%%%%%%%%%%%%%%%%%%%%%%%%%%%%%%%%%%%%%%%%%%%%%%%%%%
%%
%%   BornAgain User Manual
%%
%%   homepage:   http://www.bornagainproject.org
%%
%%   copyright:  Forschungszentrum Jülich GmbH 2015
%%
%%   license:    Creative Commons CC-BY-SA
%%
%%   authors:    Scientific Computing Group at MLZ Garching
%%               C. Durniak, M. Ganeva, G. Pospelov, W. Van Herck, J. Wuttke
%%
%%%%%%%%%%%%%%%%%%%%%%%%%%%%%%%%%%%%%%%%%%%%%%%%%%%%%%%%%%%%%%%%%%%%%%%%%%%%%%%%

\part{Physics}\label{PPHYS}

\chapter{Scattering}  \label{SSca}
\chaptermark{Scattering}


This chapter introduces the basic theory of small-angle scattering (SAS).
\index{Small-angle scattering}%
\index{SAS|see{Small-angle scattering}}%
We specifically consider scalar neutron and X-ray propagation,
adjourning the notationally more involved
theory of polarized neutrons to \Cref{SPol}.
Our exposition is self-contained,
except for the initial passage from the microscopic
to the macroscopic Schrödinger equation,
which we outline only briefly (\Cref{Swave}).
For X-rays, we obtain the same perturbed Schrödinger equation from Maxwell's equations
(\Cref{SXray}).
The standard description of scattering in first-order Born approximation
(\Cref{SBornApprox})
is introduced in a way that facilitates the following generalization
into the distorted-wave Born approximation (DWBA)
needed for grazing-incidence scattering (\Cref{SDWBA}).
This chapter finishes with a qualitative discussion
of coherence lengths (\Cref{Scoherlen}).

%%%%%%%%%%%%%%%%%%%%%%%%%%%%%%%%%%%%%%%%%%%%%%%%%%%%%%%%%%%%%%%%%%%%%%%%%%%%%%%%
\section{Neutron propagation}\label{Swave}
%%%%%%%%%%%%%%%%%%%%%%%%%%%%%%%%%%%%%%%%%%%%%%%%%%%%%%%%%%%%%%%%%%%%%%%%%%%%%%%%
\index{Wave propagation!neutrons|(}%
\index{Neutrons!wave propagation|(}%

%===============================================================================
\subsection{Coherent wave equation}\label{ScohWave}
%===============================================================================

\def\Vmac{\tilde{V}}

\index{Schrodinger@Schrödinger equation!microscopic}%
The scalar wavefunction $\psi(\r,t)$
\nomenclature[2t020]{$t$}{Time}%
\nomenclature[2r040]{$\r$}{Position}%
\nomenclature[1ψ030 2r040 2t02]{$\psi(\r,t)$}{Microscopic neutron wavefunction}%
of a free neutron
is governed by the microscopic Schrödinger equation
\begin{equation}\label{ESchrodi}
  i\hbar\partial_t \psi(\r,t)
  = \left\{-\frac{\hbar^2}{2m}\Nabla^2+V(\r)\right\} \psi(\r,t).
\end{equation}
Since BornAgain only supports elastic scattering,
\index{Elastic scattering}%
\index{Scattering!elastic}%
the potential $V(\r)$ must be time independent.
\index{Time}%
Therefore we only need to consider monochromatic waves
\index{Wave!monochromatic}%
\index{Monochromatic wave}%
with given frequency~$\omega$.
\nomenclature[1ω 020]{$\omega$}{Frequency of incident radiation}%
In consequence, the wave function
\begin{equation}\label{Estationarywave}
  \psi(\r,t) = \psi(\r)\e^{-i\omega t}
\end{equation}
\nomenclature[1ψ030 2r040 0]{$\psi(\r)$}{Stationary wavefunction}%
factorizes into a stationary wave and a time-dependent phase factor.%
\index{Stationary wave function}%
\index{Phase factor}%
\footnote
{The minus sign in the exponent of the phase factor
is an inevitable consequence of the standard form of the Schrödinger equation,
and is therefore called the \E{quantum-mechanical sign convention}.
\index{Wave propagation|seealso {Sign convention}}%
\index{Convention|see {Sign convention}}%
\index{Sign convention!wave propagation|(}%
For electromagnetic radiation
usage is less uniform.
While most optics textbooks
have adopted the quantum-mechanical convention~\cref{Estationarywave},
\index{Quantum-mechanical convention}%
in X-ray crystallography
the conjugate phase factor $\e^{+i\omega t}$ is prefered.
This \E{crystallographic sign convention}
\index{Crystallographic sign convention}%
has also been chosen
in influential texts on GISAXS (e.g.\ \cite{ReLL09}).
Here, however, we are concerned not only with X-rays,
but also with neutrons,
and therefore need to leave the Schrödinger equation~\cref{ESchrodi} intact.
Consequently, in this manual, and in the program code of BornAgain,
the quantum-mechanical sign convention~\cref{Estationarywave}
\index{Sign convention!wave propagation|)}%
is chosen.}
Inserting \cref{Estationarywave} in \cref{ESchrodi},
we obtain the stationary Schrödinger equation
\begin{equation}\label{EstatSchrodi}
  \left\{-\frac{\hbar^2}{2m}\Nabla^2+V(\r)-\hbar\omega\right\} \psi(\r) = 0.
\end{equation}
\nomenclature[2m020]{$m$}{Neutron mass}%
\nomenclature[2v130 2r040]{$V(\r)$}{Microscopic optical potential}%
\index{Potential|see {Optical potential}}%
\index{Optical potential!nuclear (microscopic)}%
The \E{nuclear} (or \E{microscopic})
\E{optical potential} $V(\r)$,
in a somewhat ``naive conception'' \cite[p.~7]{Sea89},
consists of a sum of delta functions,
representing Fermi's ``pseudopotential''.
\index{Fermi's pseudopotential}%
The superposition of the incident wave with the scattered waves
originating from each illuminated nucleus
results in \E{coherent forward scattering},
\index{Coherent forward scattering}%
in line with Huygens' principle.
\index{Huygens' principle}%

Coherent superposition also leads to \E{Bragg scattering}.
\index{Bragg scattering!by atomic lattices}%
However, Bragg scattering by atomic lattices only occurs at angles
far above the small-angle range covered in GISAS experiments.
Accordingly, it can be neglected in the analysis of GISAS data,
or at most, is taken into account as a loss channel.

Therefore,
we can neglect the atomic structure of $V(\r)$,
and perform some coarse graining to
arrive at a \E{continuum approximation}.
\index{Continuum approximation!neutron propagation}%
This is
similar to the passage from
the microscopic to the macroscopic Maxwell equations.
The details are intricate \cite{Sea89,Lax51},
but the result \cite[eq.~2.8.32]{Sea89} is very simple:
The macroscopic field equation
has still the form of a stationary Schrödinger equation,
\index{Schrodinger@Schrödinger equation!macroscopic}%
\begin{equation}\label{EmacrSchrodi}
  \left\{-\frac{\hbar^2}{2m}\Nabla^2+\Vmac(\r)-\hbar\omega\right\} \psi(\r) = 0,
\end{equation}
\nomenclature[1ψ030 2r040 2t020]{$\psi(\r,t)$}{Coherent wavefunction}%
\nomenclature[2v131 2r040]{$\Vmac(\r)$}{Macroscopic optical potential}%
where $\psi$ now stands for the \E{coherent wavefunction}
\index{Coherent wavefunction}%
\index{Wave propagation!coherent}%
obtained by superposition of
incident and forward scattered states,
and $\Vmac(\r)$ is the \E{macroscopic optical potential}.
\index{Optical potential!macroscopic}%
This potential is weak, and slowly varying compared to atomic length scales.
In the following it shall be expressed through the
\E{bound scattering length density} (SLD)
\index{Bound scattering length|see{Scattering length}}%
\index{Scattering!length density}%
\index{SLD|see{Scattering length density}}%
\cite[eq.\ 2.8.37]{Sea89},
\nomenclature[2v020 2r040]{$v(\r)$}{Scattering length density (SLD)}%
\begin{equation}
  v(\r)\coloneqq\frac{m}{2\pi \hbar^2}\Vmac(\r).
\end{equation}

%===============================================================================
\subsection{SLD fluctuations}\label{Sfluct}
%===============================================================================

In the following, we will use~\cref{EmacrSchrodi}
as a starting point to study scattering by condensed-matter samples
(we will prefer the brief term \E{sample} over \E{scattering target}).
\index{Scattering!target|see{Sample}}%
\index{Sample}%
To compute scattering cross sections from a perturbation expansion,
\index{Perturbation expansion}%
we will need to decompose the SLD as
\begin{equation}\label{Edecompose_v}
  v(\r) \coloneqq \mv(\r) + \delta v(\r).
\end{equation}
\nomenclature[1d030 2v230 2r040]{$\delta v(\r)$}{Fluctuating part of the scattering length density}%
\nomenclature[2v021]{$\mv$}{Average scattering length density}%
For $\r$ outside the finite sample volume, we require $\delta v(\r)=0$.
Inside the sample, the decomposition~\cref{Edecompose_v} is somewhat arbitrary,
and can be chosen for analytical convenience.
The macroscopic Schrödinger equation~\cref{EmacrSchrodi} becomes
\begin{equation}\label{ESchrodi3}
  \left\{ \Nabla^2 + \frac{2m\omega}{\hbar} - 4\pi\mv(\r) \right\}\psi(\r)
  = 4\pi\delta v(\r)\psi(\r).
\end{equation}
The left-hand side describes optical wave propagation,
the right-hand side is a perturbation that causes scattering.
\index{Scattering potential!neutron}%
\index{Perturbation potential|see{Scattering potential}}%
\index{Potential|see{Scattering potential}}%
Accordingly, $\mv(\r)$ should only contain SLD variations
that are simple enough to allow an analytical solution.
The SLD fluctuation~$\delta v(\r)$ then stands for the more irregular
features of a sample one ultimately wants to study in a scattering experiment.

For brevity, we rewrite~\cref{ESchrodi3} as
\Emph{
\begin{equation}\label{ESchrodiK}
  \left\{ \Nabla^2 + k(\r)^2 \right\}\psi(\r)
  = 4\pi\delta v(\r)\psi(\r).
\end{equation}
\vspace*{-5pt}}
\index{Perturbed Schrödinger equation}%
\index{Schrödinger equation!perturbed}%
This \E{perturbed Schrödinger equation} will be the starting point for all further analyses.
It contains the material-dependent wavenumber~$k(\r)$, defined by the dispersion relation
\nomenclature[2k030 2r040]{$k(\r)$}{Wavenumber}%
\index{Dispersion!neutron in homogeneous medium}%
\index{Wavenumber!material dependent}%
\begin{equation}\label{Edispersion}
  k(\r)^2 \coloneqq \frac{2m\omega}{\hbar} - 4\pi\mv(\r).
\end{equation}
Alternatively, we may write
\begin{equation}\label{Ekkn}
  k(\r)=K n(\r),
\end{equation}
using the vacuum wavenumber~$K\coloneqq 2m\omega/\hbar$
\nomenclature[2k120]{$K$}{Wavenumber in vacuum}%
and the neutron optical \E{refractive index}
\nomenclature[2n020]{$n$}{Refractive index}%
\index{Refractive index}%
\index{Index of refraction|see {Refractive index}}%
\begin{equation}\label{EnRefrIndx}
  n(\r)\coloneqq \sqrt{1-\frac{4\pi}{K^2}\mv(\r)}.
\end{equation}
It is customary to express the complex refractive index of a material
by two real parameters:
\begin{equation}\label{Endb1}
  n =  1-\delta +i\beta.
\end{equation}
\nomenclature[1δ020]{$\delta$}{Small parameter in the refractive index
   $n=1-\delta +i\beta$}%
\nomenclature[1β020]{$\beta$}{Imaginary part of the refractive index}%
For X-rays and thermal neutrons,
$\delta$ and $\beta$ are almost always nonnegative,\footnote
{The plus sign in front of~$i\beta$ is a consequence of
the quantum-mechanical sign convention;
in the X-ray crystallography convention it would be a minus sign.
\index{Refractive index!sign convention}%
\index{Sign convention!refractive index}}
and much smaller than~1.
A nonzero~$\beta$ describes absorption and leads to a damping of propagating waves.
\index{Absorption}%
\index{Damping}%

The SLD fluctuation amplitude $|\delta v(\r)|$ is at most of order $1-n$.
Therefore the right-hand side of the Schrödinger equation~\cref{ESchrodiK}
is but a weak perturbation.
This suggests a solution
by means of a perturbation expansion in powers of $\delta v$,
\index{Perturbation expansion}%
known as the \E{Born expansion} or \E{Born series}.\footnote{
Named after Max Born
who introduced it in quantum mechanics.
The idea goes back to Lord Rayleigh
who devised it for sound,
and later also applied it to electromagnetic waves,
which resulted in his famous explanation of the blue sky.}
The standard first-order Born approximation (BA, \cref{SBornApprox})
is regularly used  for the analysis of small-angle scattering (SAS) experiments.
\index{Small-angle scattering}%
\index{BA|see{Born approximation}}%
\index{Born approximation}%
For grazing-incidence small-angle scattering (GISAS),
one uses the slightly more generic distorted wave Born approximation (DWBA, \Cref{SDWBA}).
\index{Distorted-wave Born approximation}%

\index{Wave propagation!neutrons|)}%
\index{Neutrons!wave propagation|)}%

%%%%%%%%%%%%%%%%%%%%%%%%%%%%%%%%%%%%%%%%%%%%%%%%%%%%%%%%%%%%%%%%%%%%%%%%%%%%%%%%
\section{Neutron scattering in Born approximation}\label{SBornApprox}
%%%%%%%%%%%%%%%%%%%%%%%%%%%%%%%%%%%%%%%%%%%%%%%%%%%%%%%%%%%%%%%%%%%%%%%%%%%%%%%%

%===============================================================================
\subsection{The Born expansion}\label{SBornExpans}
%===============================================================================

\index{Born approximation|(}%
\index{Scattering!plane wave|(}%

To describe an elastic scattering experiment,
we need to solve the Schrödinger equation~\cref{ESchrodiK}
under the asymptotic boundary condition\footnote
{A more formal variant of this is known as the \E{Sommerfeld radiation condition}.%
\index{Sommerfeld radiation condition}%
\index{Radiation condition}%
}
\index{Boundary conditions!elastic scattering}%
\begin{equation}\label{Escabouco}
  \psi(\r)
  \simeq \psi_\si(\r) + f(\vartheta,\varphi)\frac{\e^{iKr}}{4\pi r}
  \text{~for~}r\to\infty,
\end{equation}
\nomenclature[1ψ034 2i000 2r040]{$\psi_\si(\r)$}{Incident wavefunction}%
\nomenclature[2i000]{i}{Subscript ``incident''}%
\index{Incident radiation!Born approximation}%
where $\psi_\si(\r)$ is the incident wave
as prepared by the experimental apparatus,
and the second term on the right-hand side is
the outgoing scattered wave
that carries information in form of the angular distribution
$f(\vartheta,\varphi)$.

Equation~\cref{ESchrodiK} looks
like an inhomogeneous differential equation ---
except that the right-hand side contains the unknown function~$\psi$.
The key idea of the Born expansion is to overcome this problem by iteration.
Treating the right-hand side as a given inhomogeneity,
one can solve~\cref{ESchrodiK} by the Green function method.
The incident wave~$\psi_\si(\r)$ must be determined from the homogeneous wave equation
\begin{equation}\label{EHomoK}
  \left\{\Nabla^2+k(\r)^2\right\}\psi_\si(\r) = 0.
\end{equation}
The Green function~$G(\r,\r')$ must fulfill the wave equation with an isolated inhomogeneity,
\index{Green function}%
\begin{equation}\label{EGreenK}
  \left\{\Nabla^2+k(\r)^2\right\}G(\r,\r') = \delta(\r-\r').
\end{equation}
\nomenclature[2g134]{$G(\r,\r')$}{Green function}%
Let us postpone the explicit or asymptotic solution of these two equations
to later sections (\cref{Sfarfield} for the standard plane-wave case with $k(\r)=K$,
\cref{SfarDW} for the more generic distorted-wave case).
Here we only need to assume that solutions $\psi_\si$ and~$G$ exist.
We can then posit the \E{Lippmann-Schwinger equation}
\index{Lippmann-Schwinger equation}%
\begin{equation}\label{EPsiFormal}
  \psi(\r)
  = \psi_\si(\r)
  + \int\!\d^3r'\, G(\r,\r') 4\pi\delta v(\r')\psi(\r').
\end{equation}
By operating on both sides with~$\{\Nabla^2+k(\r)^2\}$
it is easily seen that solutions~$\psi$ of this integral equation
fulfill the perturbed Schrödinger equation \cref{ESchrodiK}.

The Lippmann-Schwinger equation can be resolved into an infinite series
by iteratively substituting the full right-hand side of~\cref{EPsiFormal}
for the occurence of~$\psi$ in the integrand.
Successive terms in this series contain rising powers of $\delta v$.
Since $\delta v$ is assumed to be small, the series is likely to converge.
In \E{first-order Born approximation},
only the linear order in $\delta v$ is retained,
\begin{equation}\label{EBorn}
  \psi(\r)
  \doteq \psi_\si(\r)
  + 4\pi \int\!\d^3r'\, G(\r,\r') \delta v(\r') \psi_\si(\r').
\end{equation}
This is practically always adequate for
material investigations with X-rays or neutrons,
where the aim is to
deduce $\delta v(\r')$ from the scattered intensity ${|\psi(\r)|}^2$.
Since detectors are always placed at positions $\r$
that are not illuminated by the incident beam,
we are only interested in the scattered wave
\index{Scattered wave}%
\index{Wave!scattered}%
\begin{equation}\label{EBornS}
  \psi_\text{s}(\r)
  \coloneqq
  4\pi \int\!\d^3r'\, G(\r,\r') \delta v(\r') \psi_\si(\r').
\end{equation}
\nomenclature[1ψ034 2s000 0 2r040]{$\psi_\text{s}(\r)$}{Scattered wavefunction}%
\nomenclature[2s000 0]{s}{Subscript ``scattered''}%
For brevity and mathematical convenience,
the integral introduced in~\cref{EPsiFormal} has no bounds
and therefore formally runs over the entire space.
However, the scattering potential $\delta v(\r')$ in the integrand is nonzero
only if $\r'$ lies inside the sample volume.
On the other hand, we only need to compute the scattered wave~$\psi_\text{s}(\r)$
at detector positions~$\r$ far outside the sample.
This is just the condition for \E{Fraunhofer diffraction}:
\index{Fraunhofer approximation}%
the distance from the sample to the detector location~$\r$
must be much larger than the size of the sample.
In the context of generic scattering theory,
it is known as the \E{far-field approximation}.
\index{Far-field approximation}%
To keep notation light,
we assume from now on that the coordinate origin
\index{Coordinate origin}%
\index{Origin!coordinate system}%
lies inside the sample.
Then far-field asymptotes are obtained by taking the limit $r\to\infty$.
Since we only need the asymptote $\psi_{\text{s}\infty}(\r)$ of the scattered wave,
it is sufficient to solve~\cref{EGreenK} for the far-field Green function
\begin{equation}\label{EGinftydef}
  G_\infty(\r,\r')\coloneqq \lim_{r\to\infty} G(\r,\r').
\end{equation}
This will be done in very different ways for plane waves (\cref{Sfarfield})
and for distorted waves (\cref{SfarDW}).

\index{Scattering!plane wave|)}%
\index{Born approximation|)}%

%===============================================================================
\subsection{Vacuum solution}\label{Sfarfield}
%===============================================================================

\index{Far-field approximation|(}%

As said in connection with~\cref{Edecompose_v},
there is some freedom in the choice of $\mv(\r)$.
In the standard variant of the Born expansion,
one makes the simplest possible choice,
setting $\mv(\r)$ to the constant vacuum value.
With~\cref{Edecompose_v},
all fluctuations in $v(\r)$ are ascribed to the scattering potential~$\delta v(\r)$.
With~\cref{Edispersion}, we have $k(\r)=K$.
The homogeneous equation~\cref{EHomoK} reduces to the \E{Helmholtz equation},
\index{Helmholtz wave equation}%
\index{Wave propagation!Helmholtz equation}%
and is solved by plane waves and superpositions thereof.
In the following we choose the incident plane wave
\index{Incident radiation!plane wave}%
\begin{equation}\label{EPsi0Plane}
  \psi_\si(\r)=\e^{i \k_\si \r}
\end{equation}
\nomenclature[2k040]{$\k$}{wavevector}%
with $k_\si=K$.
Superposition of plane waves
\index{Superposition}%
will be discussed when we come to instrumental resolution effects (\cref{SBeam}).
\index{Instrument!resolution|see{Resolution}}%
\index{Resolution}%

The Green function of the inhomogeneous Helmholtz equation~\cref{EGreenK}
with $k(\r)=K$ is well known:\footnote
{Verification under the condition $\r\ne0$
is a straightforward exercise in vector analysis.
For the special case $\r=0$,
one encloses the origin in a small sphere
and integrates by means of the Gauss-Ostrogadsky divergence theorem.
This explains the appearance of the factor $4\pi$.}
\index{Green function!homogeneous material}%
\begin{equation}\label{EGreens1}
  G(\r,\r') = \frac{\e^{iK|\r-\r'|}}{4\pi |\r-\r'|}.
\end{equation}
Read as a function of~$\r$, it is an outgoing spherical wave centered at $\r'$.
To compute the far-field limit~\cref{EGinftydef},
we expand for $\r'$ with $r'\ll r$:
\begin{equation}
  \left|\r-\r'\right|
  \doteq \sqrt{r^2-2\r\,\r'}
  \doteq r - \frac{\r\,\r'}{r}
  \equiv r - \frac{\k_\sf \r'}{K},
\end{equation}
\nomenclature[2f000]{f}{Subscript ``final''}%
where we have introduced the outgoing wavevector
$  \k_\sf\coloneqq K \r / r$.
We apply this to~\cref{EGreens1},
\index{Green function!homogeneous material}%
and obtain in leading order the far-field Green function
\begin{equation}\label{EGreenFar}
  G_\infty(\r,\r')
  = \frac{\e^{iKr}}{4\pi r}\psi^*_\sf(\r'),
\end{equation}
\nomenclature[2g134 2far]{$G_\infty(\r,\r')$}{Far-field
   approximation to the Green function $G(\r,\r')$}%
where
\begin{equation}\label{EPsisfar}
  \psi_\sf(\r) \coloneqq  \e^{i\k_\sf \r}
\end{equation}
\nomenclature[1ψ034 2f000 2r040]{$\psi_\sf(\r)$}{Plane
  wave propagating from the sample towards the detector}%
is a plane wave propagating towards the detector,
and $\psi^*$ designates the complex conjugate of $\psi$.
As function of~$\r$, $G_\infty$ is an outgoing spherical wave.
Inserting \cref{EGreenFar} in \cref{EBornS},
we obtain the far-field approximation for the scattered wave,
\index{Scattered wave!far-field}%
\begin{equation}\label{EsandwichC}
  \psi_{\text{s}\infty}(\r)
  = \frac{\e^{iKr}}{r}
    \bra \psi_\si|\delta v|\psi_\sf\ket^*
\end{equation}
\nomenclature[1ψ034 2s000 2far]{$\psi_{\text{s}\infty}(\r)$}{Far-field
   approximation to the scattered wavefunction $\psi_\text{s}(\r)$}%
with the Dirac notation for the scattering matrix element
\index{Scattering matrix}%
\index{Transition matrix|see{Scattering matrix}}%
\Emph{%
\begin{equation}\label{Etrama}
  \bra \psi_\si|\delta v|\psi_\sf\ket
  \coloneqq  \int\!\d^3r\, \psi^*_\si(\r)\delta v(\r)\psi_\sf(\r).
\end{equation}
\vspace*{-5pt}}
\nomenclature[0$\langle$0]{{$\bra\ldots\vert\ldots\vert\ldots\ket$}}{Matrix
  element, defined as a volume integral}%
Since per \cref{EPsi0Plane} and~\cref{EPsisfar} our $\psi_\si$ and~$\psi_\sf$ are plane waves,
the matrix element can be simplified as
\begin{equation}\label{Echiq}
  \bra \psi_\si|\delta v|\psi_\sf\ket
  = \int\!\d^3r\, {\rm e}^{-i\k_\si\r}\delta v(\r){\rm e}^{i\k_\sf\r}
  = \int\!\d^3r\, {\rm e}^{i\q\r}\delta v(\r)
  \eqqcolon v(\q),
\end{equation}
\nomenclature[2v030 2q040]{$v(\v{q})$}{Fourier
   transform of the SLD~$\delta v(\r)$}%
with the \E{scattering vector}\footnote
{With this choice of sign,
\index{Sign convention!scattering vector}%
$\hbar\q$ is the momentum
\index{Momentum transfer|see {Scattering vector}}%
\E{gained} by the scattered neutron,
and \E{lost} by the sample.
In much of the literature the opposite convention is prefered,
since it emphasizes the sample physics over the scattering experiment.
However, when working with two-dimensional detectors
it is highly desirable to express pixel coordinates
\index{Coordinate system}
and scattering vector components
with respect to equally oriented coordinate axes,
which can only be achieved by the convention~\cref{Eq}.}
\index{Scattering!vector}%
\begin{equation}\label{Eq}
  \q\coloneqq \k_\sf-\k_\si.
\end{equation}
\nomenclature[2q040]{$\q$}{Scattering vector}%
\Cref{Echiq} summarizes the well-known fact that
small-angle neutron scattering basically measures
the Fourier transform $v(\q)$ of the SLD.
\index{Scattering potential!Fourier transform}%
\index{Fourier transform!scattering potential}%

\index{Far-field approximation|)}%


%===============================================================================
\subsection{Differential cross section}\label{SdiffCross}
%===============================================================================

In connection with \cref{EBorn} we mentioned
that a scattering experiment measures intensities~${|\psi(\r)|}^2$.
We shall now restate this in a more rigorous way.
In the case of neutron scattering,
one actually measures a \E{probability flux}.
We define it in arbitrary relative units as
\begin{equation}\label{EdefJ}
  \v{J}(\r) \coloneqq  \psi^*\frac{\Nabla}{2i}\psi - \psi\frac{\Nabla}{2i}\psi^*.
\end{equation}
\nomenclature[2j150 2r040]{$\v{J}(\r)$}{Probability flux}%
\index{Flux!incident and scattered}%
The ratio of the scattered flux hitting an infinitesimal detector area
$r^2\d\Omega$ to the incident flux is expressed as a
\E{differential cross section}
\index{Cross section}%
\index{Scattering!cross section}%
\index{Elastic scattering cross section}%
\index{Incident radiation!flux|(}%
\begin{equation}\label{Exsectiondef}
  \xElas
  \coloneqq  \frac{r^2 J(\r)}{J_\si}.
\end{equation}
\nomenclature[1ω120]{$\Omega$}{Solid angle}%
\nomenclature[1σ020]{$\sigma$}{Scattering or absorption cross section}%
% TODO RESTORE XREF
% The geometric factors that are needed to
% convert $\d\sigma/\d\Omega$ into detector counts will be discussed
% below in \cref{SdetImg}.
For a plane wave~\cref{EPsi0Plane}, the incident flux is
\index{Incident radiation!flux|)}%
\index{Flux!Born approximation}%
\begin{equation}\label{EJi}
  \v{J}_\si = \k_\si.
\end{equation}
With the far-field result~\cref{EsandwichC}
and the notation~\cref{Etrama},
the scattered flux at the detector is
\begin{equation}\label{EJr}
  \v{J}(\r)
  = \v{\hat r}\frac{K}{r^2}
    {\left|\bra\psi_\si|\delta v|\psi_\sf\ket\right|}^2.
\end{equation}
Inserting these into definition~\cref{Exsectiondef},
we obtain the generic differential cross section
of elastic scattering in first order Born approximation,
\index{Born approximation!elastic scattering cross section}%
\index{Cross section!Born approximation}%
\index{Scattering!cross section!Born approximation}%
\index{Elastic scattering cross section!Born approximation}%
\Emph{
\begin{equation}\label{Exsection}
  \xElas
  =  {\left|\bra\psi_\si|\delta v|\psi_\sf\ket\right|}^2.
\end{equation}\vspace*{-5pt}
}
As we shall see below,
it holds not only for plane waves governed
by the Helmholtz equation,
but also for distorted waves.
\index{Distorted-wave Born approximation!elastic cross section}%
In the plane-wave case \cref{Echiq} considered here,
the differential cross section is just the squared modulus
of the Fourier transform of the SLD,
\index{Scattering!length density!Fourier transform}%
\begin{equation}\label{Ecross1}
  \xElas
  = {\left| v(\q) \right|}^2.
\end{equation}

%%%%%%%%%%%%%%%%%%%%%%%%%%%%%%%%%%%%%%%%%%%%%%%%%%%%%%%%%%%%%%%%%%%%%%%%%%%%%%%%
\section{Distorted-wave Born approximation (DWBA)}\label{SDWBA}
%%%%%%%%%%%%%%%%%%%%%%%%%%%%%%%%%%%%%%%%%%%%%%%%%%%%%%%%%%%%%%%%%%%%%%%%%%%%%%%%

\index{Distorted-wave Born approximation|(}%
\index{DWBA|see {Distorted-wave Born approximation}}%

%===============================================================================
\subsection{Introduction}\label{SDWBA1}
%===============================================================================

The standard Born approximation depends on the choice $\mv=\text{const}$,
which implies that $\psi_\si$ and $\psi_\sf$ are plane waves.
In the distorted-wave Born approximation (DWBA),\footnote{
The DWBA was originally devised by Massey and Mott (ca 1933)
for collisions of charged particles.
Summaries can be found in some quantum mechanics textbooks (Messiah, Schiff)
and in monographs on scattering theory (e.~g.\ Newton).
The first explicit applications to grazing-incidence scattering
were published in 1982:
Vineyard \cite{Vin82} discussed X-ray scattering,
but failed to account for the distortion of the scattered wave;
Mazur and Mills \cite{MaMi82} deployed heavy formalism
to compute the inelastic neutron scattering cross section
of ferromagnetic surface spin waves from scratch.
A concise derivation of the DWBA cross section
was provided by Dietrich and Wagner (1984/85)
for X-rays \cite{DiWa84} and neutrons \cite{DiWa85}.
Unfortunately, their work was overlooked in much of the later literature,
which often fell back to less convincing derivations.}
this requirement is dropped.
The SLD decomposition~\cref{Edecompose_v}
is restored to full genericity,
and it is taken for granted
that somehow analytical or numerical solutions $\psi_\si$ and~$\psi_\sf$
of the homogeneous wave equation~\cref{EHomoK}
have been obtained.

Because these solutions are no longer plane waves,
we need to work out a terminological and notational distinction
that is blurred in the standard Born approximation:
we need to distinguish
 between the \E{exciting} wave~$\psi_\se$
\nomenclature[1ψ034 2e000]{$\psi_\se(\r)$}{Exciting wave}%
and the \E{incident} wave~$\psi_\si$.
The \E{exciting wave}
\index{Exciting wave}%
\index{Wave!exciting}%
is prepared far
outside the sample by a radiation source and some optical devices.
\index{Radiation source}%
It is a superposition of plane waves,
as discussed later in the context of instrumental resolution effects
(\cref{SInstr}).
Here, while discussing scattering theory,
it shall be represented by a single plane wave
$\psi_\se(\r)=\e^{i\k_\se\r}$.
This function is defined for all~$\r$,
but is physical only along the primary beam, upstream of the sample.

In contrast, by the \E{incident wave}~$\psi_\si(\r)$
\index{Incident wave!DWBA}%
\index{Incident wave!vs exciting wave}%
\index{Wave!incident}%
we understand an exact solution of~\cref{EHomoK}.
Upstream of the sample, along the primary beam, it coincides with the exciting wave.
Inside the sample, however, it undergoes refraction and reflection,
and therefore no longer is a plane wave, and no longer equals~$\psi_\se(\r)$.
This is different from the conventional Born approximation,
where $\psi_\si$ is a plane wave throughout the sample,
and therefore must not be distinguished from~$\psi_\se$.

Differently from the above derivation of the Born expansion,
we will not even attempt to obtain a full solution
\index{Green function!DWBA}%
of the Green function equation~\cref{EGreenK}.
We will use indirect arguments to directly obtain its far-field asymptote $G_\infty$.

%===============================================================================
\subsection{Reciprocity of the Green function}\label{SReci}
%===============================================================================

Our computation of~$G_\infty$ will be based on a source-detector \E{reciprocity theorem}
\index{Reciprocity|(}%
\index{Green function!reciprocity}%
for the scalar Schrödinger equation.\footnote
{There exists a confusing multitude of reciprocity theorems \cite{Pot04}.
In the end, we found it simpler to provide a derivation taylored for our application case
than to refer to the literature.}
The theorem states:
Any Green function $G(\r,\r')$
that solves~\cref{EGreenK} and as function of~$\r$ represents an outgoing wave
[in accord with the Sommerfeld radiation condition,
\index{Sommerfeld radiation condition}%
\index{Radiation condition}%
cf.~\cref{Escabouco}]
is invariant under an exchange of source location~$\rS$ and detection point~$\rD$:
\nomenclature[2r041 2d100]{$\rD$}{Position of detector}%
\nomenclature[2r041 2s100]{$\rD$}{Position of source, locus of scattering}%
\begin{equation}\label{Erecip}
  G(\rS,\rD) = G(\rD,\rS).
\end{equation}
Readers not interested in mathematical details may skip the following \E{proof}:

We introduce the auxiliary vector field
\begin{equation}
  \v{X}(\r,\rS,\rD)\coloneqq G(\r,\rD)\Nabla G(\r,\rS) - G(\r,\rS)\Nabla G(\r,\rD).
\end{equation}
%\nomenclature[2x150 2r040]{$\v{X}(\r,\rS,\rD)$}{Auxiliary vector field}%
We inscribe the sample and the detector
into a sphere $\Sphere$ around the coordinate origin with radius~$R$,
%\nomenclature[2s180]{$\Sphere$}{Auxiliary spherical volume}%
%\nomenclature[2r120]{$R$}{Radius of $\Sphere$}%
and compute the volume integral
\begin{equation}\label{Eprerecipro}
    I(\rS,\rD) \coloneqq \displaystyle\int_\Sphere\!\d^3r\,\Nabla \v{X}(\r,\rS,\rD).
\end{equation}
After a few steps, not entirely trivial, but not too difficult either,
we obtain
\begin{equation}\label{EIBD}
  I(\rS,\rD) = G(\rS,\rD) - G(\rD,\rS).
\end{equation}
Alternatively, we can compute $I$ as a surface integral
\begin{equation}
  I(\rS,\rD)
  =\displaystyle\int_{\partial\Sphere}\d\v{\sigma}\,\v{X}(\r,\rS,\rD).
\end{equation}
On the surface $\partial\Sphere$,
$G$ is an outgoing solution of the Helmholtz equation.
As such, it has a well-known series expansion in spherical coordinates
We send $R\to\infty$ so that we need only to retain the lowest order:
\begin{equation}
  \begin{array}{@{}l@{\;}c@{\;}l}
   G(\r(R,\vartheta,\varphi),\rD)
   &\doteq&\displaystyle \frac{\e^{iKR}}{4\pi R} a(\vartheta,\varphi),
   \\[3.8ex]
   G(\r(R,\vartheta,\varphi),\rS)
   &\doteq&\displaystyle \frac{\e^{iKR}}{4\pi R} b(\vartheta,\varphi).
  \end{array}
\end{equation}
The factorization of $G$ and $B$ and their common $R$ dependence imply that
\begin{equation}
  I(\rS,\rD)
  =\displaystyle\int_{\partial\Sphere}\d\sigma\,
       (\text{$R$-dependent})(ab-ba)
  = 0.
\end{equation}
Comparison with \cref{EIBD} yields \cref{Erecip},
which completes the proof.

\index{Reciprocity|)}%

%===============================================================================
\subsection{Far-field Green function}\label{SfarDW}
%===============================================================================

\def\Do{\overset{o}{D}}
\def\Go{\overset{o}{G}}
\def\pfo{\overset{o}{\psi}_\sf}

The following derivation of the far-field Green function~$G_\infty$
\index{Green function!distorted waves, far-field|(}%
is a simplified rendering of Dietrich and Wagner \cite{DiWa84},
transcribed from electromagnetic to neutron waves.
With the differential operators
\begin{equation}\label{EDDdef}
  \begin{array}{@{}l}
    \Do \coloneqq \Nabla^2+K^2, \\[1.4ex]
    D \coloneqq \Nabla^2+k(\r)^2 = \Do + \delta k(\r)^2,
  \end{array}
\end{equation}
and with $\delta k^2\coloneqq k^2-K^2$,
we rewrite the homogeneous and inhomogeneous differential equations \cref{EHomoK}
and~\cref{EGreenK} for plane waves and distorted waves as
\begin{equation}
  \begin{array}{l@{\hspace{2em}}l}
    \Do\pfo  (\r) = 0, &\Do\Go(\r,\r') = \delta(\r-\r'), \\[1.4ex]
    D\psi_\sf(\r) = 0, &D  G  (\r,\r') = \delta(\r-\r').
  \end{array}
\end{equation}
We now posit an integral equation that determines
the full Green function $G$ if the vacuum Green function~$\Go$ is given:
\begin{equation}\label{EGoG}
  \Go(\rD,\rS) = G(\rD,\rS) + \int\!\d^3r\, \Go(\rD,\r)\delta k(\r)^2 G(\r,\rS).
\end{equation}
It can be verified by operating on both sides with $\Do$.\footnote
{This is not entirely convincing:
Equality of the second derivates is not sufficient to prove equality of
two sides in an equation.
For a full prove, we also have to show that the two sides, and their first derivatives,
agree at least in one point~$\rD$.
Perhaps this point must be chosen in the limit~$r_\text{D}\to\infty$.
Our source~\cite{DiWa84} provides no clue.}
We use the reciprocity~\cref{Erecip} to transform~\cref{EGoG} into
\begin{equation}
  \Go(\rD,\rS) = G(\rD,\rS) + \int\!\d^3r\, G(\rD,\r)\delta k(\r)^2 \Go(\r,\rS).
\end{equation}
Below, we will only need the far-field limit in $\rD$:
\begin{equation}\label{EGoFF}
  \Go_\infty(\rD,\rS) = G_\infty(\rD,\rS) + \int\!\d^3r\, G_\infty(\rD,\r)\delta k(\r)^2 \Go(\r,\rS).
\end{equation}
We also posit an integral equation that determines the distorted wave~$\pfo$
if the plane wave~$\psi_\sf$ is given:
\begin{equation}\label{Epfo}
  \pfo(\rS) = \psi_\sf(\rS) + \int\!\d^3r\, \psi_\sf(\r)\delta k(\r)^2 \Go(\r,\rS).
\end{equation}
Verification can again be done by operating on both sides with~$\Do$.
We recall the far-field asymptote of the vacuum Green function from~\cref{EGreenFar},
and rewrite it as
\begin{equation}
  \Go_\infty(\rD,\rS) = \phi(\rD)\pfo^*(\rS)
\end{equation}
with the far-field prefactor
\begin{equation}\label{Effphidef}
  \phi(r) \coloneqq \frac{\e^{iKr}}{4\pi r}.
\end{equation}
We take complex conjugate of both sides of~\cref{Epfo},
 and multiply them with $\phi(\rD)$ to obtain
\begin{equation}\label{Epfo2}
  \Go_\infty(\rD,\rS)
  = \phi(\rD)\psi^*_\sf(\rS) + \int\!\d^3r\, \phi(\rD)\psi^*_\sf(\r)\delta k(\r)^2 \Go(\r,\rS).
\end{equation}
We compare with~\cref{EGoFF},
and read off the far-field asymptote of the distorted-waves Green function
\begin{equation}
  G_\infty(\rD,\rS) = \phi(\rD)\psi^*_\sf(\rS).
\end{equation}
Note that this is formally identical with with \cref{EGreenFar},
but no longer requires $\psi_\sf$ to be a plane wave.
\index{Green function!distorted waves, far-field|)}%

\Emph{Accordingly,
the differential cross section is still given by \cref{Exsection},
and the scattering matrix element by~\cref{Etrama}.}
\index{Scattering matrix!distorted waves}%
\index{Cross section!distorted-wave Born approximation}%
\index{Scattering!cross section!distorted-wave Born approximation}%
\index{Elastic scattering cross section!distorted-wave Born approximation}%
The plane-wave form~\cref{Echiq}, however, does no longer hold.
Its replacement depends on the distorted wavefunctions
$\psi_\si$ and $\psi_\sf$,
is therefore application specific,
and will be worked out later (\cref{Swave21}).

\index{Distorted-wave Born approximation|)}%

%%%%%%%%%%%%%%%%%%%%%%%%%%%%%%%%%%%%%%%%%%%%%%%%%%%%%%%%%%%%%%%%%%%%%%%%%%%%%%%%
\section{X-ray scattering}\label{SXray}
%%%%%%%%%%%%%%%%%%%%%%%%%%%%%%%%%%%%%%%%%%%%%%%%%%%%%%%%%%%%%%%%%%%%%%%%%%%%%%%%
\index{X-ray!scattering theory|(}%

%===============================================================================
\subsection{Wave equation}\label{SXwave}
%===============================================================================

The propagation of X-rays is governed by Maxwell's equations,
\index{Maxwell's equations}%
\index{Wave propagation!X-rays}%
\begin{equation}\label{EMaxwell}
  \begin{array}{@{}l@{\quad}l@{\quad}l}
    \Nabla\times\v{E}=-\partial_t \v{B},
   &\Nabla\v{B}=0,
   &\v{B}=\v{\mu}(\r)\mu_0\v{H},
   \\[2ex]
    \Nabla\times\v{H}=+\partial_t \v{D},
   &\Nabla\v{D}=0,
   &\v{D}=\v{\eps}(\r)\eps_0\v{E}.
  \end{array}
\end{equation}
\nomenclature[2e150 2r040 2t020]{$\v{E}(\r,t)$}{Electric field}%
\nomenclature[2h150 2r040 2t020]{$\v{H}(\r,t)$}{Magnetizing field}%
\nomenclature[2d150 2r040 2t020]{$\v{D}(\r,t)$}{Displacement field}%
\nomenclature[2b150 2r040 2t020]{$\v{B}(\r,t)$}{Magnetic field}%
\nomenclature[1ε070 2r040]{$\v\eps(\r)$}{Relative dielectric permittivity tensor}%
\nomenclature[1μ070 2r040]{$\v\mu(\r)$}{Relative magnetic permeability tensor}%
\nomenclature[1ε024 00]{$\eps_0$}{Vacuum permittivity, 8.854\ldots As/Vm}%
\nomenclature[1μ024 00]{$\mu_0$}{Vacuum permeability, 1.256\ldots Vs/Am}%
Since BornAgain only addresses elastic scattering,
\index{Elastic scattering}%
\index{Scattering!elastic}%
we assume the permeability and permittivity tensors $\v\mu$ and~$\v\eps$
to be time-independent.
\index{Time}%
Therefore, as in~\cref{ScohWave}, we only need to consider monochromatic waves
\index{Wave!monochromatic}%
\index{Monochromatic wave}%
with given frequency~$\omega$,
and each of the fields $\v{E}$, $\v{D}$, $\v{H}$, $\v{B}$
factorizes into a stationary field and a time-dependent phase factor.
\index{Phase factor}%
In the following, we will concentrate on the electric field
\index{Electric field}%
\begin{equation}
  \v{E}(\r,t) = \v{E}(\r)\e^{-i\omega t}.
\end{equation}
The other three fields can be obtained from~$\v{E}$
by straightforward application of~\cref{EMaxwell}.

Since magnetic refraction or scattering is beyong the scope of BornAgain,
the relative magnetic permeability tensor is always $\v{\mu}(\r)=1$.
\index{Permeability}%
\index{Magnetic permeability}%
As customary in SAXS and GISAXS,
\index{Grazing-incidence small-angle scattering!dielectric model}%
\index{Small-angle scattering!dielectric model}%
we assume
that the dielectric properties of the material are those of a polarizable electron cloud.\footnote
{This is occasionally called the \E{Laue model}
\index{Laue model}%
 \cite{Lau31}.}
Thereby the relative dielectric permittivity tensor~$\v{\eps}$
\index{Dielectric permittivity}%
\index{Permittivity}%
becomes a scalar,
\begin{equation}
  \eps(\r)=1-\frac{4\pi r_e}{K^2}\rho(\r),
\end{equation}
\nomenclature[1ε030 2r040]{$\eps(\r)$}{Relative dielectric permittivity function}%
\nomenclature[1ρ030 2r040]{$\rho(\r)$}{Electron number density}%
with the classical electron radius~$r_e=e^2/mc^2\simeq2.8\cdot10^{-15}$~m,
\index{Electron radius}%
\index{Classical electron radius}%
\nomenclature[2r024 2e000]{$r_e$}{Classical electron radius~$2.817\ldots^{-15}$~m}%
the electron number density~$\rho(\r)$,
\index{Electron density}%
\index{Density!electron}%
\index{Number density|see{Density}}%
and the vacuum wavenumber~$K$,
given by the dispersion relation
\begin{equation}
  K^2 = \mu_0\eps_0\omega^2.
\end{equation}
\index{Dispersion!X-rays}%
With these simplifying assumptions about $\v{\eps}$ and~$\v{\mu}$,
Maxwell's equations yield the wave equation
\begin{equation}\label{ENabNabE}
  \Nabla\times\Nabla\times\v{E} = K^2\eps(\r)\v{E}.
\end{equation}
\index{Wave equation!X-rays}%
\index{X-ray!wave equation}%
To solve this equation by a perturbation expansion,
\index{Perturbation expansion}%
we proceed as in \cref{Sfluct},
and decompose the dielectric permittivity
into a slowly varying component and a perturbation that fluctuates on atomic scales,
\begin{equation}
  \eps(\r) = \overline{\eps}(\r) + \delta\eps(\r).
\end{equation}
\nomenclature[1ε031 2r040]{$\overline\eps(\r)$}{Slowly varying part of the permittivity~$\eps(\r)$}%
\nomenclature[1δ030
  1ε030 2r040]{$\delta\eps(\r)$}{Fast varying part of the permittivity~$\eps(\r)$}%
In analogy to~\cref{Ekkn} we define the material-dependent wavenumber by
\index{Wavenumber!material dependent}%
\begin{equation}
  k(\r)^2 \coloneqq K^2 \overline{\eps}(\r).
\end{equation}
Furthermore, we define the scattering potential by
\index{Scattering potential!X-ray}%
\begin{equation}
  4\pi\delta v(\r) \coloneqq - K^2\delta\eps.
\end{equation}
The wave equation~\cref{ENabNabE} then becomes
\begin{equation}\label{EwaveE3}
  \left\{-\Nabla\times\Nabla\times\ONE + k(\r)^2\right\}\v{E}(\r)
  = 4\pi\delta v(\r)\v{E}(\r).
\end{equation}
\index{Wave equation!X-rays}%
\index{X-ray!wave equation}%
This is very similar to the perturbed Schrödinger equation~\cref{ESchrodiK}.
\index{Perturbed Schrödinger equation}%
\index{Schrödinger equation!perturbed}%
There are only two differences:
the more complicated differential operator,
and the fact that $\v{E}(\r)$ is vector valued,
whereas unpolarized neutrons are described by a scalar wave function~$\psi(\r)$.

%===============================================================================
\subsection{Solution}\label{SXscasol}
%===============================================================================

\def\TD{\TENS{D}}
\def\TG{\TENS{G}}
\def\TDo{\TENS{\overset{o}{D}}}
\def\TGo{\TENS{\overset{o}{G}}}
\def\Ef{\v{E}_\sf}
\def\Efo{\TENS{\overset{o}{\v{E}}}_\sf}
\def\Efoc{\Efo^*}

The far-field Green function for the wave equation~\cref{EwaveE3}
can be derived along the lines of~\Cref{SfarDW}
except that differential operators and Green functions are now dyadic tensors \cite{DiWa84}.
\Work{This will be worked out soon.}
As result, the scattering matrix element~\cref{Etrama} of the scalar case
needs to be multiplied with \E{polarization factors}.
\index{Polarization!factor (X-ray)}%
For GISAS geometry, they can be found in equations (92)--(95) of Ref.~\cite{ReLL09}.
For the usual small scattering angles of GISAS,
they are close to unity and therefore can be neglected.
In BornAgain, they are not yet implemented.

\iffalse
\index{Green function!distorted waves, far-field|(}%
Instead of \cref{EDDdef}, we will need the differential operators
\begin{equation}
  \begin{array}{@{}l}
    \TDo \coloneqq -\Nabla\times\Nabla\times\ONE+K^2
          = \Nabla^2-\Nabla\cdot\Nabla+K^2, \\[1.4ex]
   \TD \coloneqq -\Nabla\times\Nabla\times\ONE+k(\r)^2 = \TDo + \delta k(\r)^2,
  \end{array}
\end{equation}
with $\delta k^2\coloneqq k^2-K^2$ as before.
Based on~\cref{EwaveE3},
the homogeneous and inhomogeneous differential equations
for plane waves and distorted waves are
\begin{equation}
  \begin{array}{l@{\hspace{2em}}l}
    \TDo\Efo  (\r) = 0, &\TDo\TGo(\r,\r') = \ONE\delta(\r-\r'), \\[1.4ex]
    \TD\Ef(\r) = 0, &\TD  \TG  (\r,\r') = \ONE\delta(\r-\r').
  \end{array}
\end{equation}
\Cref{EGoFF} becomes
\begin{equation}\label{EXGoFF}
  \TGo_\infty(\rD,\rS)
  = \TG_\infty(\rD,\rS) + \int\!\d^3r\, \TG_\infty(\rD,\r)\delta k(\r)^2 \TGo(\r,\rS).
\end{equation}
\Cref{Epfo} gets replaced by
\begin{equation}\label{EXpfo}
  \Efo(\rS) = \Ef(\rS) + \int\!\d^3r\, \Ef(\r)\delta k(\r)^2 \TGo(\r,\rS).
\end{equation}
These equations can be verified by operating on both sides with~$\TDo$.
We recall the far-field asymptote of the vacuum Green function from~\cref{EGreenFar},
and rewrite it as
\begin{equation}
  \TGo_\infty(\rD,\rS) = \phi(\rD)\Efoc(\rS)
\end{equation}
with the far-field prefactor $\phi$ as defined in~\cref{Effphidef}.
We take complex conjugate of both sides of~\cref{EXpfo},
 and multiply them with $\phi(\rD)$ to obtain
\begin{equation}\label{EXpfo2}
  \TGo_\infty(\rD,\rS)
  = \phi(\rD)\psi^*_\sf(\rS) + \int\!\d^3r\, \phi(\rD)\psi^*_\sf(\r)\delta k(\r)^2 \TGo(\r,\rS).
\end{equation}
We compare with~\cref{EXGoFF},
and read off the far-field asymptote of the distorted-waves Green function
\begin{equation}
  \TG_\infty(\rD,\rS) = \phi(\rD)\psi^*_\sf(\rS).
\end{equation}
Note that this is formally identical with with \cref{EGreenFar},
but no longer requires $\Ef$ to be a plane wave.
\index{Green function!distorted waves, far-field|)}%

\Emph{Accordingly,
the differential cross section is still given by \cref{Exsection},
and the scattering matrix element by~\cref{Etrama}.}
\index{Scattering matrix!distorted waves}%
\index{Cross section!distorted-wave Born approximation}%
\index{Scattering!cross section!distorted-wave Born approximation}%
\index{Elastic scattering cross section!distorted-wave Born approximation}%
The plane-wave form~\cref{Echiq}, however, does no longer hold.
Its replacement depends on the distorted wavefunctions
$\psi_\si$ and $\Ef$,
is therefore application specific,
and will be worked out later (\cref{Swave21}).
\fi
%% TODO \index{Distorted-wave Born approximation|)}%

\index{X-ray!scattering theory|)}%

%%%%%%%%%%%%%%%%%%%%%%%%%%%%%%%%%%%%%%%%%%%%%%%%%%%%%%%%%%%%%%%%%%%%%%%%%%%%%%%%
\section{Coherent vs incoherent scattering}\label{Scoherlen}
%%%%%%%%%%%%%%%%%%%%%%%%%%%%%%%%%%%%%%%%%%%%%%%%%%%%%%%%%%%%%%%%%%%%%%%%%%%%%%%%

%===============================================================================
\subsection{Coherence length}
%===============================================================================

Per \cref{Exsection} and~\cref{Etrama},
\index{Coherence length|(}%
the matrix element $\bra \psi_\si|\delta v|\psi_\sf\ket$
is given by a three-dimensional integral
\begin{equation}\label{Etrama3}
  \bra \psi_\si|\delta v|\psi_\sf\ket
  \coloneqq  \int\!\d^3r\, \psi^{*}_\si(\r)\delta v(\r)\psi_\sf(\r).
\end{equation}
The integration domain is effectively limited to a finite $z$ interval,
where $\delta v(\r)$ is nonzero.
The horizontal integration domain, however, is infinite
within our formalism,
which is of course an idealization.
Obviously, physical integration limits are imposed by the finite
\index{Sample area}%
\E{illuminated sample area}.\footnote
{We assume a well aligned instrument,
for which the beam footprint and the backtracked detector footprint
\index{Illumination!beam footprint on sample}%
\index{Beam footprint}%
\index{Backtracking!beam footprint}%
agree within reasonable accuracy.}
Another limitation comes from the finite \E{coherence length}
of the instrumental setup,
which usually is much shorter than the sample width and length
%This is of importance in neutron scattering
%where typical sample dimensions of 1\ldots10~mm
%are much larger than the relevant coherence length,
%which is of the order 10\ldots100~$\upmu$m
\cite{HaPR10,MaMM14}.\footnote
{These two references also make clear that
  the theoretical description and the experimental determination of
  coherence lengths are difficult problems and subject of ongoing research.}

While each single neutron is described by a wavefunction
that allows for \E{coherent} superposition of
different contributions to the scattered wavefunction,
the final detector statistics
\index{Detector statistics}%
is given by an \E{incoherent} sum
over the differential cross sections of individual neutrons.
The finite \E{resolution}
\index{Resolution|(}%
of an experimental setup is in part due to the fact that
different neutrons have different wavenumbers,
originate\footnote
{It is reasonable to take the last collision in the moderator
  as the \E{origin} of a neutron ray,
  since collisions between neutrons and hydrogen nuclei bound in
  disordered matter lead to almost perfect decoherence.}
at different points in the moderator,
and are detected at slightly different points within one detector pixel.
This can be modeled by computing expected scattering intensities as
averages over different neutrons with
$K$, $\v{\hat k}_\si$, and $\v{\hat k}_\sf$ drawn at random
from appropriate distributions.
% TODO RESTORE TEMPORARILY REMOVED XREF as described in \cref{Sresolution}.

However, this is not the full story.
In the above introduction to the Born approximation
we have made the standard assumption
that an incoming neutron can be described by a plane wave
$\psi_\si=\e^{i\k_\si\r}$.
The wavefunction $\psi_\sf$ traced back from the detector is also
approximated by a plane wave.
In the DWBA we allow these waves to be distorted within the sample,
but when impinging on the sample they still are plane.
A plane wave obviously is an idealized concept,
since it has infinite lateral extension.
The \E{transverse coherence length} indicates the scale
beyond which this approximation becomes invalid.
At larger scales, the wave fronts appear randomly distorted.
Physical causes of these distortions include
reflections in the neutron guide,
diffraction by guide windows and other slits,
and diffraction by imperfect monochromator crystals.
Of course the distorted wave still admits a Fourier decomposition
into plane waves with slightly different wavevectors.
In practice, it is impossible to distinguish this spread of wavevectors
from the incoherent spread described in the previous paragraph.
The instrumental resolution function therefore
accounts for both causes of wavevector distortion.
\index{Resolution|)}%

Usually, therefore, a GISANS image is an incoherent average
over coherent diffraction patterns collected from
many small subareas of the sample.
Only horizontal sample structures on scales smaller the coherence length
yield interference patterns.
Structure fluctuations on larger scales
produce said incoherent average of different GISANS images.

The crossover from coherent to incoherent scattering is of course
a gradual one.
The coherence length indicates where a certain, somewhat arbitrary degree
of decoherence is reached.
Under these reservations
one defines a \E{coherence spot}
in the cross section of an approximately plane wave
as an area where the coherence is above a certain threshold.
Unless the wave has been prepared in a highly anisotropic guide and slit system,
this spot is about circular.
Under grazing incidence conditions however,
the projection of this spot onto the sample surface
yields a very elongated ellipse.
Therefore, the coherence length is much larger in $x$ than
in $y$ or $z$ direction.\footnote
{This has nothing to do with the distinction of
  \E{transverse} and \E{longitudinal} coherence length.
  Longitudinal coherence has to do with wavelength stability
  and is of no importance for elastic scattering.
  We are talking here about \E{horizontal} and \E{vertical}
  projections of the \E{transverse} coherence length.}

%===============================================================================
\subsection{Implementation}
%===============================================================================


\Note{\indent Unless otherwise said, \BornAgain\ simulates
  \E{coherent} diffraction patterns obtained by
  the linear superposition of scattered waves.
  To simulate an \E{incoherent} mixture of diffraction patterns,
  the most generic solution is a script with an outer loop
  that averages over several coherent computations with
  appropriately distributed parameters.}

\Warn{\indent Currently, \BornAgain\ does not support interferences
  between particles in different layers.}
% TODO: more about implementation !

\index{Coherence length|)}%
