%%%%%%%%%%%%%%%%%%%%%%%%%%%%%%%%%%%%%%%%%%%%%%%%%%%%%%%%%%%%%%%%%%%%%%%%%%%%%%%%
%%
%%   BornAgain User Manual
%%
%%   homepage:   http://www.bornagainproject.org
%%
%%   copyright:  Forschungszentrum Jülich GmbH 2015
%%
%%   license:    Creative Commons CC-BY-SA
%%
%%   authors:    Scientific Computing Group at MLZ Garching
%%               C. Durniak, M. Ganeva, G. Pospelov, W. Van Herck, J. Wuttke
%%
%%%%%%%%%%%%%%%%%%%%%%%%%%%%%%%%%%%%%%%%%%%%%%%%%%%%%%%%%%%%%%%%%%%%%%%%%%%%%%%%

\chapter{Embedded particles}\label{SRefParticles}

%%%%%%%%%%%%%%%%%%%%%%%%%%%%%%%%%%%%%%%%%%%%%%%%%%%%%%%%%%%%%%%%%%%%%%%%%%%%%%%%
\section{Particle layout}\label{SRefPLay}
%%%%%%%%%%%%%%%%%%%%%%%%%%%%%%%%%%%%%%%%%%%%%%%%%%%%%%%%%%%%%%%%%%%%%%%%%%%%%%%%

A \ttIdx{ParticleLayout} is constructed through the following API:

\setCpp
\begin{lstlisting}
class ParticleLayout {
    ParticleLayout();
    ParticleLayout(const IAbstractParticle& particle);
    ParticleLayout(const IAbstractParticle& particle, double abundance);
    void addParticle(const IAbstractParticle& particle);
    void addParticle(const IAbstractParticle& particle, double abundance);
    void addParticle(const IParticle& particle, double abundance, const kvector_t position);
    void addParticle(const IParticle& particle, double abundance, const kvector_t position, const IRotation& rotation);
    void addInterferenceFunction(const IInterferenceFunction& interference_function);
    void setTotalParticleSurfaceDensity(double particle_density);
};
\end{lstlisting}

An \ttIdx{IAbstractParticle} is either an \ttIdx{IParticle} (\cref{SRefIPart})
or a \ttIdx{Particle\-Distri\-bution} (\cref{SRefPDis}).
\index{Particle!IAbstractParticle@\Code{IAbstractParticle}}%
\index{Particle!IParticle@\Code{IParticle}}%
\index{Particle!ParticleDistribution@\Code{ParticleDistribution}}%
In any case, it does not stand for \E{one} particle,
but for one kind of particles.
A \texttt{ParticleLayout} can be loaded with any number of \texttt{IAbstractParticle}s,
using the function \clFct{IAbstractParticle}{addParticle}.

Most often, a \texttt{ParticleLayout} is to be loaded with just one \texttt{IAbstractParticle}.
In this case, \texttt{IAbstractParticle} can be passed as an argument of the
\texttt{ParticleLayout} constructor,
and no call of \texttt{addParticle} is needed;
by default, the \texttt{abundance} is set to~1.
\index{Abundance}%
If there are several \texttt{IAbstractParticle}s,
then the parameter \texttt{abundance} of the function \texttt{addParticle} must be used for
a number that expresses the relative number density of the particle kind.

\Work{Explain \texttt{position} and \ttIdx{IRotation} \ldots}

%%%%%%%%%%%%%%%%%%%%%%%%%%%%%%%%%%%%%%%%%%%%%%%%%%%%%%%%%%%%%%%%%%%%%%%%%%%%%%%%
\section{IParticle}\label{SRefIPart}
%%%%%%%%%%%%%%%%%%%%%%%%%%%%%%%%%%%%%%%%%%%%%%%%%%%%%%%%%%%%%%%%%%%%%%%%%%%%%%%%

\MissingSection

%%%%%%%%%%%%%%%%%%%%%%%%%%%%%%%%%%%%%%%%%%%%%%%%%%%%%%%%%%%%%%%%%%%%%%%%%%%%%%%%
\section{Particle distribution}\label{SRefPDis}
%%%%%%%%%%%%%%%%%%%%%%%%%%%%%%%%%%%%%%%%%%%%%%%%%%%%%%%%%%%%%%%%%%%%%%%%%%%%%%%%

\MissingSection
